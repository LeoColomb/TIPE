%
% Travail d'Initiative Personnelle Encadré
% https://github.com/LeoColomb/TIPE
% Copyright (c) 2014 Léo Colombaro
% Licensed under MIT License
%

% -----------------
% -- Définition  --
% -----------------

% Format du papier
\documentclass[11pt,a4paper]{report}

% Langue
\usepackage[utf8]{inputenc}
\usepackage[francais]{babel}
\usepackage[T1]{fontenc}

% Métadata
\title{Fractales attractives}
\author{Léo Colombaro}
\date{2013\space\textperiodcentered\space 2014}

% Modules primaires
\usepackage{url} % URL
\usepackage{hyperref} % Lien internes
\usepackage{graphicx} % Images
\usepackage{wrapfig} % Images intégrées
\usepackage{verbatim} % Texte brut
\usepackage{moreverb} % Encodage
%\usepackage{quote} % Citations
\usepackage{listings} % Syntaxe programme
\usepackage{color} % Coloration

% -----------------
% -- Affichage   --
% -----------------

% Présentation
\usepackage[pdfspacing]{classicthesis}
\usepackage{lmodern}
\usepackage{fourier}

% Alinéa
\setlength{\parindent}{pt}

% Marges
\usepackage[left=3cm,right=3cm,top=3cm,bottom=3cm]{geometry}

% Sections
%\renewcommand\thechapter{\Roman{chapter}}
\titlespacing{\section} {5ex} {5ex} {2ex} {}
\renewcommand*\thesection{\arabic{section}}
\titlespacing{\subsection} {10ex} {2ex} {1ex} {}
\renewcommand\thesubsection{\alph{subsection}}

% Chapitre sans saut de page
\newcommand{\chapterspace}[1]{\begingroup
  \let\clearpage\relax
  \vspace{1em}
  \chapter{#1}
  \endgroup}

% Table des matieres
\usepackage{tocloft}
\renewcommand{\cftpartleader}{\cftdotfill{\cftdotsep}}
\renewcommand{\cftchapleader}{\cftdotfill{\cftdotsep}}
\renewcommand{\cftsecleader}{\cftdotfill{\cftdotsep}}
\renewcommand{\cftsubsecleader}{\cftdotfill{\cftdotsep}}

% -----------------
% -- Maths       --
% -----------------

% Modules Mathématiques
\usepackage{amsmath}
\usepackage{amsfonts}
\usepackage{amssymb}
\usepackage{mathrsfs}
\newcommand*\E{\ensuremath{\textup{e}}\xspace}

% Ensembles
\newcommand{\N}{\mathbb{N}}
\newcommand{\Z}{\mathbb{Z}}
\newcommand{\Q}{\mathbb{Q}}
\newcommand{\R}{\mathbb{R}}
\newcommand{\C}{\mathbb{C}}

% Intervalles
\newcommand{\intval}[4]{\mathopen{#1}#2
  \mathclose{}\mathpunct{};#3
  \mathclose{#4}}
\newcommand{\intvalff}[2]{\intval{[}{#1}{#2}{]}}
\newcommand{\intvalof}[2]{\intval{]}{#1}{#2}{]}}
\newcommand{\intvalfo}[2]{\intval{[}{#1}{#2}{[}}
\newcommand{\intvaloo}[2]{\intval{]}{#1}{#2}{[}}

% Valeur absolue / Normes
\newcommand{\abs}[1]{\left\lvert#1\right\rvert}
\newcommand{\norme}[1]{\left\lVert#1\right\rVert}

% --------------
% -- Document --
% --------------

\begin{document}

% Contenu
%
% Travail d'Initiative Personnelle Encadré
% https://github.com/LeoColomb/TIPE
% Copyright (c) 2014 Léo Colombaro
% Licensed under MIT License
%

\begin{titlepage}

% Page de garde
\makeatletter
  \begin{flushright}
      {\textbf {\huge \@title}}\\
    \vspace{3em}
      {\Large Travail d'Initiative Personnelle Encadré}\\
      Classe Préparatoire aux Grandes Écoles\\
      Mathématiques Physique et Sciences de l'Ingénieur\\
      École Nationale de Chimie Physique Biologie de Paris\\
    \vspace{3em}
      {\large \@author}\\
      {\small \@date \space --- v0.0.1}
  \end{flushright}
\makeatother

\vfill

% Résumé
%\begin{abstract}
Maecenas dictum aliquet risus eget adipiscing. Integer enim enim, posuere a tempor quis, sagittis vel dolor. Mauris luctus egestas neque viverra ultricies. Sed accumsan cursus ligula, ac elementum tortor fermentum fringilla. Vivamus pharetra est ut nisi iaculis tristique. Donec vehicula eget metus nec fermentum. Praesent at metus nec arcu vestibulum consequat eu condimentum quam. Aenean ornare eros leo, id pellentesque sem vulputate ac. Curabitur a congue erat. Praesent fermentum malesuada condimentum. Pellentesque molestie nulla nulla. Suspendisse in lectus tristique, hendrerit metus ac, luctus quam. Vestibulum vulputate augue justo, vel hendrerit lectus elementum eu. Quisque pellentesque turpis gravida, porta nisi sed, egestas arcu. Nunc a porta velit.
%\end{abstract}

\end{titlepage}

% Table des matières
\tableofcontents

%
% Travail d'Initiative Personnelle Encadré
% https://github.com/LeoColomb/TIPE
% Copyright (c) 2014 Léo Colombaro
% Licensed under MIT License
%

\section{Suite logistique}
\subsection{Définition et notions préliminaires}
Pour commencer, on prend selon la définition de la suite logistique
par récurrence, $ x_{n+1} = \mu x_n(1 - x_n) $, où l'on prend 
\begin{itemize}
\item $\mu \in \left[0,4\right] $, réel nommé le curseur
\item $ x_{0} \in \left]0,1\right[ $, réel nommé le graine
\end{itemize}

\subsection{Modèle continu}
On préfèrera rapidement se ramener à la fonction qui est associée à cette suite,
puisque qu'elle permet une étude plus poussée, notamment grâce aux ressources mathématiques
d'analyse sur une fonction continue et dérivable.
\subsection{Étude comportementale selon le curseur}
La particularité de cette suite est son comportement selon $ \mu $,
d'où l'on tire par une étude de cas des valeurs particulières.
\subsection{Théorème fondamental}
Il est alors nécessaire de définir le théorème qui nous permet de 
caractériser le comportement de la suite au voisinage d'un point fixe,
c'est-à-dire s'il est attractif ou répulsif. La démonstration se
repose sur l'inégalité des accroissements finis.

%
% Travail d'Initiative Personnelle Encadré
% https://github.com/LeoColomb/TIPE
% Copyright (c) 2014 Léo Colombaro
% Licensed under MIT License
%

\section{Attracteurs}
\subsection{Définition}
Par définition, c'est un réel dont s'approchent indéfiniment
une infinité de valeurs d'une suite. En application à la suite logistique,
on détermine alors des cycles de récurrence.
\subsection{Itérés de $ f $ associée à la suite logistique}
Ces derniers cycles mettent en avant des propriétés sur l'itération de
la fonction (i.e. sur $ k $ avec $ f^k = f \circ \ldots \circ f $),
puisque les valeurs des abscisses des intersections entre ces
cycles et la courbe ne sont rien d'autre que les abscisses des points fixes
de l'itération de $f$ suivante.
\subsection{Attracteurs étranges}
On peut aussi chercher à augmenter de dimension, c'est-à-dire
prendre la fonction $ f $ sur deux (ou plus) paramètres. On peut
alors obtenir des représentations originales telles que l'attracteur
de Hénon.

%
% Travail d'Initiative Personnelle Encadré
% https://github.com/LeoColomb/TIPE
% Copyright (c) 2014 Léo Colombaro
% Licensed under MIT License
%

\chapter{Le chaos}

%
% Travail d'Initiative Personnelle Encadré
% https://github.com/LeoColomb/TIPE
% Copyright (c) 2014 Léo Colombaro
% Licensed under MIT License
%
  \begin{center}
  ---------
  \end{center}
\section*{Références}
\begin{itemize}
\item \textbf{Daniel Perrin}, \textit{La suite logistique et le chaos}, IREM, 2008
\item \textbf{Douglas Hofstadter}, \textit{Jeux Mathématiques}, Pour la Science, 1982
\item \textbf{Elmer G. Wiens}, \textit{Nonlinear Dynamics}, www.egwald.ca
\item \textbf{Mitchell Feigenbaum}, \textit{Universal behavior in nonlinear systems}, Los Alamos Science, 1980
\item \textbf{Robert Ferréol}, \textit{Construire des fractals grâce aux AFC}, Tangente, 2004
\item \textbf{Elisabeth Busser}, \textit{Excursion au pays des attracteurs étranges}
\end{itemize}

\appendix



%\printindex

% Glossaire
%\setglossarystyle{altlist}
%\printglossary

\end{document}
