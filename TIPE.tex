%
% Travail d'Initiative Personnelle Encadré
% https://github.com/LeoColomb/TIPE
% Copyright (c) 2014 Léo Colombaro
% Licensed under MIT License
%

% -----------------
% -- Définition  --
% -----------------

% Format du papier
\documentclass[11pt,a4paper]{report}

% Définissions de langue
\usepackage[utf8]{inputenc}
\usepackage[francais]{babel}
\usepackage[T1]{fontenc}

% Métadata
\title{La suite logistique et le chaos}
\author{Léo Colombaro}
\date{2013\space\textperiodcentered\space 2014}

% -----------------
% -- Paramétrage --
% -----------------

% Style Présentation
\usepackage[pdfspacing]{classicthesis}
\usepackage{lmodern}
\usepackage{fourier}
\usepackage[nottoc,numbib]{tocbibind}
\usepackage[toc]{glossaries}

% Alinéa
\setlength{\parindent}{0pt}

% Marges
\usepackage[left=4cm,right=4cm,top=4cm,bottom=4cm]{geometry}

% Définition des sections - sous-sections
\titlespacing{\section} {5ex} {5ex} {2ex} {}
\renewcommand*\thesection{\arabic{section}}
\titlespacing{\subsection} {10ex} {2ex} {1ex} {}
\renewcommand\thesubsection{\alph{subsection}}

% Chapitre sans saut de page
\newcommand{\chapterspace}[1]{\begingroup
  \let\clearpage\relax
  \vspace{10em}
  \chapter{#1}
  \endgroup}

% Modules URL & Code
\usepackage{url}
\usepackage{graphicx} % Required for including pictures
\usepackage{wrapfig} % Allows in-line images
\usepackage{verbatim}
\usepackage{moreverb}

% -----------------
% -- Maths       --
% -----------------

% Modules Mathématiques
\usepackage{amsmath}
\usepackage{amsfonts}
\usepackage{amssymb}
\usepackage{mathrsfs}
\newcommand*\E{\ensuremath{\textup{e}}\xspace}

% Ensembles
\newcommand{\N}{\mathbb{N}}
\newcommand{\Z}{\mathbb{Z}}
\newcommand{\Q}{\mathbb{Q}}
\newcommand{\R}{\mathbb{R}}
\newcommand{\C}{\mathbb{C}}

% Intervalles
\newcommand{\intval}[4]{\mathopen{#1}#2
  \mathclose{}\mathpunct{};#3
  \mathclose{#4}}
\newcommand{\intvalff}[2]{\intval{[}{#1}{#2}{]}}
\newcommand{\intvalof}[2]{\intval{]}{#1}{#2}{]}}
\newcommand{\intvalfo}[2]{\intval{[}{#1}{#2}{[}}
\newcommand{\intvaloo}[2]{\intval{]}{#1}{#2}{[}}

% Valeur absolue - Normes
\newcommand{\abs}[1]{\left\lvert#1\right\rvert}
\newcommand{\norme}[1]{\left\lVert#1\right\rVert}

% --------------
% -- Document --
% --------------

\begin{document}

% Contenu
%
% Travail d'Initiative Personnelle Encadré
% https://github.com/LeoColomb/TIPE
% Copyright (c) 2014 Léo Colombaro
% Licensed under MIT License
%

\begin{abstract}
   Votre résumé commence ici...
\end{abstract}


%
% Travail d'Initiative Personnelle Encadré
% https://github.com/LeoColomb/TIPE
% Copyright (c) 2014 Léo Colombaro
% Licensed under MIT License
%

\chapter{Suite logistique}
\section{Définition et notions préliminaires}
\section{Théorèmes fondamentaux}
\section{Étude comportementale}
\section{Application}

\cite{Perrin}
\cite{Attract}

\newglossaryentry{computer}
{
  name=computer,
  description={is a programmable machine that receives input,
               stores and manipulates data, and provides
               output in a useful format}
}

%%
% Travail d'Initiative Personnelle Encadré
% https://github.com/LeoColomb/TIPE
% Copyright (c) 2014 Léo Colombaro
% Licensed under MIT License
%

\chapter{Suite logistique}
\section{Définition et notions préliminaires}
\section{Théorèmes fondamentaux}
\section{Étude comportementale}
\section{Application}

\cite{Perrin}
\cite{Attract}

\newglossaryentry{computer}
{
  name=computer,
  description={is a programmable machine that receives input,
               stores and manipulates data, and provides
               output in a useful format}
}


% Glossaire
\makeglossaries

% Bibliographie
\bibliographystyle{unsrt}
\bibliography{../src/refs}

\end{document}
