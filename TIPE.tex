%
% Travail d'Initiative Personnelle Encadré
% https://github.com/LeoColomb/TIPE
% Copyright (c) 2014 Léo Colombaro
% Licensed under MIT License
%

% -----------------
% -- Définition  --
% -----------------

% Format du papier
\documentclass[11pt,a4paper]{report}

% Langue
\usepackage[utf8]{inputenc}
\usepackage[francais]{babel}
\usepackage[T1]{fontenc}

% Métadata
\title{Fractales attractives}
\author{Léo Colombaro}
\date{2013\space\textperiodcentered\space 2014}

% Modules primaires
\usepackage{url} % URL
\usepackage{hyperref} % Lien internes
\usepackage{graphicx} % Images
\usepackage{wrapfig} % Images intégrées
\usepackage{verbatim} % Texte brut
\usepackage{moreverb} % Encodage
%\usepackage{quote} % Citations
\usepackage{listings} % Syntaxe programme
\usepackage{color} % Coloration

% -----------------
% -- Affichage   --
% -----------------

% Présentation
\usepackage[pdfspacing]{classicthesis}
\usepackage{lmodern}
\usepackage{fourier}

% Alinéa
\setlength{\parindent}{pt}

% Marges
\usepackage[left=3cm,right=3cm,top=3cm,bottom=3cm]{geometry}

% Sections
%\renewcommand\thechapter{\Roman{chapter}}
\titlespacing{\section} {5ex} {5ex} {2ex} {}
\renewcommand*\thesection{\arabic{section}}
\titlespacing{\subsection} {10ex} {2ex} {1ex} {}
\renewcommand\thesubsection{\alph{subsection}}

% Chapitre sans saut de page
\newcommand{\chapterspace}[1]{\begingroup
  \let\clearpage\relax
  \vspace{1em}
  \chapter{#1}
  \endgroup}

% Table des matieres
\usepackage{tocloft}
\renewcommand{\cftpartleader}{\cftdotfill{\cftdotsep}}
\renewcommand{\cftchapleader}{\cftdotfill{\cftdotsep}}
\renewcommand{\cftsecleader}{\cftdotfill{\cftdotsep}}
\renewcommand{\cftsubsecleader}{\cftdotfill{\cftdotsep}}

% -----------------
% -- Maths       --
% -----------------

% Modules Mathématiques
\usepackage{amsmath}
\usepackage{amsfonts}
\usepackage{amssymb}
\usepackage{mathrsfs}
\newcommand*\E{\ensuremath{\textup{e}}\xspace}

% Ensembles
\newcommand{\N}{\mathbb{N}}
\newcommand{\Z}{\mathbb{Z}}
\newcommand{\Q}{\mathbb{Q}}
\newcommand{\R}{\mathbb{R}}
\newcommand{\C}{\mathbb{C}}

% Intervalles
\newcommand{\intval}[4]{\mathopen{#1}#2
  \mathclose{}\mathpunct{};#3
  \mathclose{#4}}
\newcommand{\intvalff}[2]{\intval{[}{#1}{#2}{]}}
\newcommand{\intvalof}[2]{\intval{]}{#1}{#2}{]}}
\newcommand{\intvalfo}[2]{\intval{[}{#1}{#2}{[}}
\newcommand{\intvaloo}[2]{\intval{]}{#1}{#2}{[}}

% Valeur absolue / Normes
\newcommand{\abs}[1]{\left\lvert#1\right\rvert}
\newcommand{\norme}[1]{\left\lVert#1\right\rVert}

% --------------
% -- Document --
% --------------

\begin{document}

% Contenu
%
% Travail d'Initiative Personnelle Encadré
% https://github.com/LeoColomb/TIPE
% Copyright (c) 2014 Léo Colombaro
% Licensed under MIT License
%

\begin{abstract}
   Votre résumé commence ici...
\end{abstract}


%
% Travail d'Initiative Personnelle Encadré
% https://github.com/LeoColomb/TIPE
% Copyright (c) 2014 Léo Colombaro
% Licensed under MIT License
%

\chapter{Suite logistique}
\section{Définition et notions préliminaires}
\section{Théorèmes fondamentaux}
\section{Étude comportementale}
\section{Application}

\cite{Perrin}
\cite{Attract}

\newglossaryentry{computer}
{
  name=computer,
  description={is a programmable machine that receives input,
               stores and manipulates data, and provides
               output in a useful format}
}

%
% Travail d'Initiative Personnelle Encadré
% https://github.com/LeoColomb/TIPE
% Copyright (c) 2014 Léo Colombaro
% Licensed under MIT License
%

\chapter{La notion d'attracteur}

%
% Travail d'Initiative Personnelle Encadré
% https://github.com/LeoColomb/TIPE
% Copyright (c) 2014 Léo Colombaro
% Licensed under MIT License
%

\chapter{Le chaos}

%
% Travail d'Initiative Personnelle Encadré
% https://github.com/LeoColomb/TIPE
% Copyright (c) 2014 Léo Colombaro
% Licensed under MIT License
%
  \begin{center}
  ---------
  \end{center}
\section*{Références}
\begin{itemize}
\item \textbf{Daniel Perrin}, \textit{La suite logistique et le chaos}, IREM, 2008
\item \textbf{Douglas Hofstadter}, \textit{Jeux Mathématiques}, Pour la Science, 1982
\item \textbf{Elmer G. Wiens}, \textit{Nonlinear Dynamics}, www.egwald.ca
\item \textbf{Mitchell Feigenbaum}, \textit{Universal behavior in nonlinear systems}, Los Alamos Science, 1980
\item \textbf{Robert Ferréol}, \textit{Construire des fractals grâce aux AFC}, Tangente, 2004
\item \textbf{Elisabeth Busser}, \textit{Excursion au pays des attracteurs étranges}
\end{itemize}

\appendix



%\printindex

% Glossaire
%\setglossarystyle{altlist}
%\printglossary

\end{document}
