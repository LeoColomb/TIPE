%
% Travail d'Initiative Personnelle Encadré
% https://github.com/LeoColomb/TIPE
% Copyright (c) 2014 Léo Colombaro
% Licensed under MIT License
%

\section{Attracteurs}
\subsection{Définition}
Par définition, c'est un réel dont s'approchent indéfiniment
une infinité de valeurs d'une suite. En application à la suite logistique,
on détermine alors des cycles de récurrence.
\subsection{Itérés de $ f $ associée à la suite logistique}
Ces derniers cycles mettent en avant des propriétés sur l'itération de
la fonction (i.e. sur $ k $ avec $ f^k = f \circ \ldots \circ f $),
puisque les valeurs des abscisses des intersections entre ces
cycles et la courbe ne sont rien d'autre que les abscisses des points fixes
de l'itération de $f$ suivante.
\subsection{Attracteurs étranges}
On peut aussi chercher à augmenter de dimension, c'est-à-dire
prendre la fonction $ f $ sur deux (ou plus) paramètres. On peut
alors obtenir des représentations originales telles que l'attracteur
de Hénon.
