%
% Travail d'Initiative Personnelle Encadré
% https://github.com/LeoColomb/TIPE
% Copyright (c) 2014 Léo Colombaro
% Licensed under MIT License
%

\chapter{Suite logistique}
\section{Définition et notions préliminaires}
\subsection{Introduction}
On s'aperçoit rapidement de la diversité des utilisations possibles
de la suite logistique. L'origine de l'étude des systèmes dynamique est 
motivé par ces applications concrètes qui permettent d'expliquer des
mécanisme naturels. C'est ainsi que nous le rappelle Daniel \cite{Perrin}.
\begin{quotation}
Le type même de problèmes qui est à l'origine de la théorie des systèmes
dynamiques est celui de l'évolution d'une population en fonction du temps.
Le mot population est à prendre ici en un sens très large. Il peut aussi bien
s'agir d'une population humaine, qu'animale, des victimes d'une épidémie,
d'un ensemble de molécules, de particules, etc. Les modèles dont je vais parler
sont des modèles déterministes. Cela signifie qu'ils sont régis par une loi
bien déterminée, qui doit permettre, en théorie, de décrire leur évolution à
partir d'un état initial connu. On sait depuis Poincaré que, malgré cette 
hypothèse restrictive, le comportement de ces modèles peut être excessivement
compliqué, en particulier à cause de la sensibilité du système aux conditions
initiales.
\end{quotation}
L'importance des conditions initiales est en effet immense, puisque le
comportement y est déterminé.\\
Nous nous proposons donc dans cette première partie d'étudier l'influence
des conditions initiales sur la suite logistique, pour comprendre l'intérêt
applicatif de cette suite.

\subsection{Principe}
\subsubsection{Modèle continu}

\subsubsection{Modèle discret}
Tout d'abord, il convient de définir précisément la suite logistique et les
objets qui la composent.\\
Nommons-la $(x_n)_{n\in \N}$. On donne l'expression par récurrence :
\begin{Large}
\begin{align*}
x_{n+1} = \mu x_n(1 - x_n)
\end{align}
\end{Large}
avec $\mu $ un réel positif, nommé, et $ x_{0} $ le premier terme de la liste.
\section{Théorèmes fondamentaux}
\section{Étude comportementale}
\subsection{Cas $0 < \mu < 1$}
\subsection{Cas $\mu = 1$}
\subsection{Cas $1 < \mu < 2$}
\subsection{Cas $\mu = 2$}
\subsection{Cas $2 < \mu < 3$}
\subsection{Cas $\mu = 3$}
\section{Application}

