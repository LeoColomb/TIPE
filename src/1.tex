%
% Travail d'Initiative Personnelle Encadré
% https://github.com/LeoColomb/TIPE
% Copyright (c) 2014 Léo Colombaro
% Licensed under MIT License
%

\section{Suite logistique}
\subsection{Définition et notions préliminaires}
Pour commencer, on prend selon la définition de la suite logistique
par récurrence, $ x_{n+1} = \mu x_n(1 - x_n) $, où l'on prend 
\begin{itemize}
\item $\mu \in \left[0,4\right] $, réel nommé le curseur
\item $ x_{0} \in \left]0,1\right[ $, réel nommé le graine
\end{itemize}

\subsection{Modèle continu}
On préfèrera rapidement se ramener à la fonction qui est associée à cette suite,
puisque qu'elle permet une étude plus poussée, notamment grâce aux ressources mathématiques
d'analyse sur une fonction continue et dérivable.
\subsection{Étude comportementale selon le curseur}
La particularité de cette suite est son comportement selon $ \mu $,
d'où l'on tire par une étude de cas des valeurs particulières.
\subsection{Théorème fondamental}
Il est alors nécessaire de définir le théorème qui nous permet de 
caractériser le comportement de la suite au voisinage d'un point fixe,
c'est-à-dire s'il est attractif ou répulsif. La démonstration se
repose sur l'inégalité des accroissements finis.
